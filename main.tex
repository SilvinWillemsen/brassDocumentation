\documentclass[dvipsnames]{article}
\usepackage[utf8]{inputenc}
\usepackage[left=3cm, right=3cm, top=2cm]{geometry}
\title{Brass}
\author{Silvin Willemsen}
\date{March 2020}

\usepackage{natbib}
\usepackage{graphicx}
\usepackage{appendix}
\usepackage{amsmath}
\usepackage{amsfonts}
\usepackage{amssymb}

\usepackage{xcolor}
\def\SBcomment[#1]{\textcolor{red}{#1}}
\def\SWcomment[#1]{\textcolor{blue}{#1}}
\def\SScomment[#1]{\textcolor{green}{#1}}
\def\type[#1]{\textcolor{purple}{#1}}

\def\dxp{\delta_{x+}}
\def\dxm{\delta_{x-}}
\def\mup{\mu_{x+}}
\def\mum{\mu_{x-}}
\def\Sp{S_{l+1/2}}
\def\Sm{S_{l-1/2}}
\def\Psilp{\Psi_{l+1}^n}
\def\Psilm{\Psi_{l-1}^n}
\def\Psinp{\Psi_l^{n+1}}
\def\Psinm{\Psi_l^{n-1}}
\def\Psiln{\Psi_l^n}
\def\Sbar{\bar S_l}
\def\z{z^{-1}}
\def\muTT{\mu_{tt}}
\def\dtd{\delta_{t\cdot}}
\begin{document}
\maketitle

\section{Introduction}
This document shows the work done and documentation on the brass part of the PhD.

Main references are \cite{Bilbao2009} and \cite{Bilbao2018}.

\section{Variable cross-section}
\subsection{Preamble}
Schemes with variable cross-section behave best (have the greatest band-width) when stretched coordinates are used. Following Section 5.3 of \cite{Bilbao2009}, we can devine a new coordinate $\alpha = \alpha(x)$ where the map $x\rightarrow\alpha$ is ``smooth and one-to-one" (i.e. if $x_1\leq x_2$ then $\alpha_1\leq \alpha_2$). Partial derivatives are as follows:
\begin{equation}
   \partial_x \quad\longrightarrow\quad \alpha'\partial_\alpha \qquad \text{and} \qquad \partial_x^2\quad \longrightarrow\quad\alpha'\partial_\alpha\left(\alpha'\partial_\alpha\right),
\end{equation}
with $\alpha'= \partial \alpha/\partial x$ which describes the rate-of-change of grid spacing with respect to the original grid spacing. In discrete time, a centered approximation for the second spatial derivative is
\begin{equation}\label{eq:secondOrderVarying}
    \delta_{xx} \quad \longrightarrow \quad \alpha'\delta_{\alpha+}\big((\mu_{\alpha-}\alpha')\delta_{\alpha-}\big).
\end{equation}
This operator applied to a grid function $u_l^n$ and expanded yields
\begin{equation}
    \frac{\alpha'_l}{2h^2} \Big((\alpha'_{l+1}+\alpha'_l)u_{l+1}^n + (\alpha'_l+\alpha'_{l-1})u_{l-1}^n - (\alpha'_{l+1}+2\alpha'_l+\alpha'_{l-1})u_l^n\Big),
\end{equation}
or
\begin{equation}
    \frac{\alpha'_l}{h^2}\Big((\mu_{\alpha+}\alpha_l')u_{l+1}^n+(\mu_{\alpha-}\alpha_l')u_{l-1}^n-2(\mu_{\alpha\alpha}\alpha'_l)u_l^n\Big)
\end{equation}
\subsubsection{String of varying cross-section}
The PDE of an ideal string with a varying cross-section is
\begin{equation}\label{eq:idealVarying}
    \epsilon^2\partial_t^2u=c_0^2\partial^2_xu,
\end{equation}
where $c_0^2 = T_0/\rho A_0$ is the reference wave-speed (in m/s) and $\epsilon = \epsilon(x)$ is a spatially varying factor (in ?unit?). Using Eq. \eqref{eq:secondOrderVarying} where $ \alpha'\rightarrow\epsilon$ we can rewrite Eq. \eqref{eq:idealVarying} to
\begin{align}
    \epsilon^2\partial_t^2u&=c_0^2\epsilon\partial_\alpha(\epsilon\partial_\alpha u),\\
    \epsilon\partial_t^2u&=c_0^2\partial_\alpha(\epsilon\partial_\alpha u),
\end{align}

% The obvious finite difference scheme for this would be
% \begin{equation}
%     \epsilon^2(x)\delta_{tt}u=c_0^2\delta_{xx}u.
% \end{equation}
% U
\subsection{Webster's equation}
The first main difference between the 1D brass PDE and the 1D wave equation is the possibility of having a variable cross-section. Following Section 19.3 from \cite{Bilbao2018}, the PDE for a 1D (axially symmetric) acoustic tube with variable cross-section is (also known as \textit{Webster}'s equation)
\begin{equation}\label{eq:webstersPDE}
    S\partial_t^2\Psi = c^2\partial_x(S\partial_x\Psi),
\end{equation}
with \textit{acoustic potential} $\Psi = \Psi(x,t)$ (m$^2$/s), $S = S(x)$ is the cross sectional area (m$^2$) and wave speed $c$ (m/s).

\subsection{Discretisation}
Introducing interleaved gridpoints at $n-1/2$ and $n+1/2$ for $S$, a we can discretise Eq. \eqref{eq:webstersPDE} (following \cite{Bilbao2018}) to
\begin{equation}\label{eq:discWebster}
    \Sbar \delta_{tt}\Psi^n_l = c^2\dxp(\Sm(\delta_{x-}\Psiln)),
\end{equation}
where
\begin{equation}
    \Sbar = \mu_{t+}\Sm = \frac{\Sp + \Sm}{2}.
\end{equation}
The right side of the equation in \eqref{eq:discWebster} contains an operator applied to two grid functions ($S$ and $\Psi$) multiplied onto each other. In order to expand this, we need to use the product rule (Eq. (2.23) in \cite{Bilbao2009}) which is
\begin{equation}
    \dxp (u_lw_l) = (\dxp u_l)(\mup{w_l}) + (\mup u_l)(\dxp w_l).
\end{equation}
In the case of \eqref{eq:discWebster}, $u_l \triangleq \Sm$ and $w_l \triangleq \dxm\Psiln$. Expanding (retaining the notation for $\Sbar$) and solving for $\Psinp$ yields (Appendix \ref{app:webstersUpdateEq})
\begin{equation}
    \Psinp = 2(1-\lambda^2)\Psiln-\Psinm+ \frac{\lambda^2\Sp}{\Sbar}\Psilp + \frac{\lambda^2\Sm}{\Sbar}\Psilm,\label{eq:webstersUpdateEq}
\end{equation}
which is identical to Eq. (19.51) in \cite{Bilbao2018}.

\subsection{Boundary Conditions}
The choices for boundary conditions in an acoustic tube are open and closed, defined as \cite{Bilbao2018}
\begin{equation}
    \begin{split}
        \partial_t\Psi &= 0\ \text{(open, Dirichlet)}\\
        \partial_x\Psi &= 0\ \text{(closed, Neumann)},
    \end{split}
\end{equation} 
at the ends of the tube. This might be slightly counter-intuitive as in the case of a string ``closed" might imply the ``clamped" or Dirichlet boundary condition. The opposite can be intuitively shown imagining a wave front with a positive acoustic potential moving through a tube and hitting a closed end. What comes back is also a wave front with a positive acoustic potential, i.e., the sign of the potential does not flip, which also happens using the free or Neumann condition for the string.

In this case we follow \cite[Chapter 9]{Bilbao2009} and use the following
\begin{equation}\label{eq:openClosed}
    \partial_x\Psi(0, t) = 0 \quad \text{and} \quad \partial_t\Psi(L, t) = 0
\end{equation}
i.e. closed at the left end and open at the right end. In discrete time we have two choices for the closed condition
\begin{equation}\label{eq:centNonCentBound}
\begin{split}
    \delta_{x\cdot}\Psi_0^n &= 0 \ \Rightarrow \ \Psi_{-1}^n = \Psi_1^n \quad \text{(centered)}\\
    \delta_{x-}\Psi_0^n &= 0\  \Rightarrow \ \Psi_{-1}^n = \Psi_0^n\quad \text{(non-centered)}
\end{split}
\end{equation}
At the left boundary we can now solve Eq. \eqref{eq:webstersUpdateEq} for the centered case:
\begin{equation}
    \begin{aligned}
        \Psi_0^{n+1} &= 2(1-\lambda^2)\Psi_0^n-\Psi_0^{n-1}+ \frac{\lambda^2S_{1/2}}{\bar S_0}\Psi_1^n + \frac{\lambda^2S_{-1/2}}{\bar S_0}\Psi_{-1}^n\nonumber\\
        \Psi_0^{n+1} &= 2(1-\lambda^2)\Psi_0^n-\Psi_0^{n-1}+ \frac{\lambda^2(S_{1/2}+S_{-1/2})}{\bar S_0}\Psi_1^n\nonumber\\
         \Psi_0^{n+1} &= 2(1-\lambda^2)\Psi_0^n-\Psi_0^{n-1}+ 2\lambda^2\Psi_1^n,
    \end{aligned}
\end{equation}
and the non-centered case
\begin{equation}\label{eq:nonCentLeft}
    \Psi_0^{n+1} = 2(1-\lambda^2)\Psi_0^n-\Psi_0^{n-1}+ \frac{\lambda^2S_{1/2}}{\bar S_0}\Psi_1^n + \frac{\lambda^2S_{-1/2}}{\bar S_0}\Psi_0^n.
\end{equation}
As can be seen from the equations above, we need undefined points $\bar S_0$ and $S_{-1/2}$. At the left boundary, we set $\bar S_0 = S_0$ from which, we can calculate $S_{-1/2}$:
\begin{equation}
        S_0 = \frac{1}{2}(S_{1/2} + S_{-1/2}) \ \Rightarrow \  S_{-1/2}
        = 2S_0 - S_{1/2}
\end{equation}
The same can be done for the right boundary ($\bar S_N = S_N$) if this is chosen to be anything else but open (e.g., closed or radiating -- see Section \ref{sec:radiating}):
\begin{equation}
    S_N = \frac{1}{2}(S_{N+1/2} + S_{N-1/2}) \ \Rightarrow \ S_{N+1/2} = 2S_N - S_{N-1/2}.
\end{equation}
For now though, we follow the conditions given in \eqref{eq:openClosed} and we can simply set the right boundary to its initial state
\begin{equation}
    \Psi_N^n = \Psi_N^0
\end{equation}
which is normally $0$. A more realistic open end is a radiating one, which can be found below.
\subsubsection{Radiating end}\label{sec:radiating}
We can change the condition presented in Eq. \eqref{eq:openClosed} to a radiating end,
\begin{equation}\label{eq:radCont}
    \partial_x\Psi(L,t) = -a_1\partial_t\Psi(L,t)-a_2\Psi(L,t)
\end{equation}
where \cite{Bilbao2009}
\begin{equation}
    a_1 = \frac{1}{2(0.8216)^2c} \quad \text{and} \quad a_2 = \frac{L}{0.8216\sqrt{S_0S(1)/\pi}}.
\end{equation}
taken from \cite{Atig2004} and are valid for a tube terminating on an infinite plane. The terms in Eq. \eqref{eq:radCont} are a damping and an inertia term where $a_1$ is a loss coefficient and $a_2$ is the \textbf{inertia coefficient}. The centered and non-centered case are defined as
\begin{equation}\label{eq:rightBoundaryConditions}
\begin{split}
    \delta_{x\cdot}\Psi_N^n &= 0 \ \Rightarrow \ \Psi_{N+1}^n = \Psi_{N-1}^n \quad \text{(centered)}\\
    \delta_{x+}\Psi_N^n &= 0\  \Rightarrow \ \Psi_{N+1}^n = \Psi_N^n\qquad \text{(non-centered)}
\end{split}
\end{equation}
First, we solve Eq. \eqref{eq:radCont} for the centered (Eq. (9.16) in \cite{Bilbao2009})
\begin{equation}\label{eq:centRadBound}
    \delta_{x\cdot}\Psi_N^n = -a_1\dtd\Psi_N^n - a_2\mu_{t\cdot}\Psi_N^n
\end{equation}
which can be expanded and solved for $\Psi_{N+1}^n$ according to
\begin{align}
    \frac{1}{2h}(\Psi_{N+1}^n - \Psi_{N-1}^n) &= -\frac{a_1}{2k}(\Psi_N^{n+1} - \Psi_N^{n-1}) - \frac{a_2}{2}(\Psi_N^{n+1} + \Psi_N^{n-1})\nonumber\\
    \Psi_{N+1}^n &= h\left(-\frac{a_1}{k}(\Psi_N^{n+1} - \Psi_N^{n-1}) - a_2(\Psi_N^{n+1} + \Psi_N^{n-1})\right) + \Psi_{N-1}^n,
\end{align}
which can be substituted into Eq. \eqref{eq:webstersUpdateEq} (Appendix \ref{app:centeredRad}) 
\begin{equation}
    \Psi_N^{n+1} = \frac{2(1-\lambda^2)\Psi_N^n-\Psi_N^{n-1}+\frac{h\lambda^2S_{N+1/2}}{\bar S_N}\left(\frac{a_1}{k}-a_2\right)\Psi_N^{n-1} + 2\lambda^2\Psi_{N-1}^n}{\left(1+\left(\frac{a_1}{k}+a_2\right)\frac{h\lambda^2S_{N+1/2}}{\bar S_N}\right)}.
\end{equation}
The same can be done for the non-centered case (Eq. (9.15) in \cite{Bilbao2009})
\begin{equation}\label{eq:nonCentRadBound}
    \delta_{x+}\Psi_N^n = -a_1\dtd\Psi_N^n - a_2\mu_{t\cdot}\Psi_N^n
\end{equation}
which when solved for $\Psi_{N+1}^n$ yields
\begin{align}
    \frac{1}{h}(\Psi_{N+1}^n - \Psi_{N}^n) &= -\frac{a_1}{2k}(\Psi_N^{n+1} - \Psi_N^{n-1}) - \frac{a_2}{2}(\Psi_N^{n+1} + \Psi_N^{n-1})\nonumber\\
        \Psi_{N+1}^n &= h\left(-\frac{a_1}{2k}(\Psi_N^{n+1} - \Psi_N^{n-1}) - \frac{a_2}{2}(\Psi_N^{n+1} + \Psi_N^{n-1})\right) + \Psi_{N}^n.
\end{align}
Substituted into Eq. \eqref{eq:webstersUpdateEq} yields (Appendix \ref{app:nonCentRad})
\begin{equation}
    \Psi_N^{n+1} = \frac{2(1-\lambda^2)\Psi_N^n-\Psi_N^{n-1}+\frac{h\lambda^2S_{N+1/2}}{\bar S_N}\left(\frac{a_1}{2k}-\frac{a_2}{2}\right)\Psi_N^{n-1} + \frac{\lambda^2S_{N+1/2}}{\bar S_N}\Psi_{N}^n + \frac{\lambda^2S_{N-1/2}}{\bar S_N}\Psi_{N-1}^n}{\left(1+\left(\frac{a_1}{2k}+\frac{a_2}{2}\right)\frac{h\lambda^2S_{N+1/2}}{\bar S_N}\right)}.
\end{equation}

\subsection{Energy}
In continuous time, the rate of change of the energy of a PDE can be obtained by taking the inner product with the first-order time derivative of the state. Using state $\Psi$, this yields
\begin{equation}
    \frac{d\mathfrak{H}}{dt} = \langle \partial_t\Psi, \text{PDE} \rangle\quad \text{where} \quad \mathfrak{H} = \mathfrak{T} + \mathfrak{D},
\end{equation}
with total energy (or Hamiltonian) $\mathfrak{H}$, kinetic energy $\mathfrak{T}$ and potential energy $\mathfrak{D}$.

In discrete time, the energy of a FDS can be obtained by taking the inner product the equation with $\dtd\Psiln$ according to 
\begin{equation}\label{eq:energyForm}
    \delta_{t+}\mathfrak{h} = \langle \dtd\Psiln,\text{FDS}\rangle \quad \text{where} \quad \mathfrak{h} = \mathfrak{t} + \mathfrak{v},
\end{equation}
where $\mathfrak{h}$, $\mathfrak{t}$ and $\mathfrak{v}$ are the discrete counterparts of $\mathfrak{H}$, $\mathfrak{T}$ and $\mathfrak{D}$.

With a boundary term present, Eq. \eqref{eq:energyForm} becomes
\begin{equation}\label{eq:energyFormBoundary}
    \delta_{t+}\mathfrak{h} = \mathfrak{b}.
\end{equation}
In the case of the radiating boundary -- with an inertia and damping term -- the energy equation is of the form
\begin{equation}\label{eq:radiatingEnergyForm}
    \delta_{t+}(\mathfrak{h}+\mathfrak{h}_\text{b}) = -\mathfrak{q}_\text{b},
\end{equation}
where $\mathfrak{h}_\text{b}$ is the energy stored by the boundary and $\mathfrak{q}_\text{b}$ is the loss energy.
\subsubsection{Kinetic Energy}
The (discrete) kinetic energy of the system $\mathfrak{t}$ of Webster's equation in Eq. \eqref{eq:discWebster} is found by taking the inner product of the left side of the equation with $\dtd\Psi$ according to
\begin{equation}
    \delta_{t+}\mathfrak{t} = \langle\dtd\Psiln,\Sbar \delta_{tt}\Psiln  \rangle_\mathcal{D},
\end{equation}
where domain $\mathcal{D}\in [0, N]$. As $\Sbar$ is merely a coefficient here, we can use identity (2.22a) from \cite{Bilbao2009} and rewrite to
\begin{equation}
    \begin{aligned}
    \delta_{t+}\left(\frac{1}{2}\sum_\mathcal{D}h\Sbar(\delta_{t-}\Psiln)^2\right)\quad \Rightarrow \quad \delta_{t+}\left(\frac{1}{2}\sum_\mathcal{D}h(\sqrt{\Sbar}\delta_{t-}\Psiln)^2\right)
    \end{aligned}.
\end{equation}
We place $\Sbar$ as a square-root in the equation, so that we can rewrite it as a norm over a domain as
\begin{equation}
    \mathfrak{t} = \frac{1}{2}\lVert\sqrt{\bar S_l}\delta_{t-}\Psiln\rVert_\mathcal{D}^2
\end{equation}
just like Eq. (9.14) in \cite{Bilbao2009}. When the boundaries of the scheme are centered we need to use a primed inner product (Eq. (5.23) in \cite{Bilbao2009}) as Bilbao explains right below Eq. (5.28) in \cite{Bilbao2009}. However, in the spatially varying case (which we are dealing with here), a more general weighted inner product needs to be used, which for a domain $\mathcal{D} = [0,\hdots,N]$ is defined as (as given in Eq. (5.38) in \cite{Bilbao2009}) 
\begin{equation}\label{eq:weightedInnerProduct}
    \langle f,g \rangle_{\mathcal{D}}^{\epsilon_\text{l}, \epsilon_\text{r}}= \sum_{l = 1}^{N-1}hf_lg_l + \frac{\epsilon_\text{l}}{2}hf_0g_0 + \frac{\epsilon_\text{r}}{2}hf_Ng_N,
\end{equation}
where free parameters $\epsilon_\text{l}, \epsilon_\text{r} > 0$ `tune' the weighting of the boundaries. For the kinetic energy, we now have
\begin{equation}
    \mathfrak{t} = \frac{1}{2}\left(\lVert\sqrt{\bar S_l}\delta_{t-}\Psiln \rVert_\mathcal{D}^{\epsilon_\text{l}, \epsilon_\text{r}}\right)^2,
\end{equation}
which, when expanded, yields
\begin{align}
    \mathfrak{t} &= \frac{1}{2}\left(\sqrt{\sum_{l=1}^{N-1}h\left(\sqrt{\bar S_l}\delta_{t-}\Psiln\right)^2 + \frac{\epsilon_\text{l}}{2}h\left(\sqrt{\bar S_0 }\delta_{t-}\Psi_0^n\right)^2 + \frac{\epsilon_\text{r}}{2}h\left(\sqrt{\bar S_N }\delta_{t-}\Psi_N^n\right)^2}\right)^2\nonumber\\
     &=\frac{1}{2}\left(\sum_{l=1}^{N-1}h\bar S_l(\delta_{t-}\Psiln)^2 + \frac{\epsilon_\text{l}}{2}h\bar S_0 (\delta_{t-}\Psi_0^n)^2 + \frac{\epsilon_\text{r}}{2}h\bar S_N (\delta_{t-}\Psi_N^n)^2\right)    .
\end{align}
The values of $\epsilon_\text{l}$ and $\epsilon_\text{r}$ can be calculated by performing the same energy analysis techniques on the right side of the equation to obtain the potential energy and the boundary terms, and will be given below.


\subsubsection{Potential Energy}
It can be shown that \eqref{eq:discWebster} is equal to
\begin{equation}
    \Sbar \delta_{tt}\Psi^n_l = c^2\dxm\big(\Sp(\dxp\Psiln)\big),
\end{equation}
i.e. changing the signs of the operators at the left side of the equation. Using $(\mup S = S_{l+1/2})$, we define for the non-centered case
\begin{equation}\label{eq:potContEnergy}
c^2\langle \dtd\Psiln, \dxm((\mup S)(\dxp\Psiln))\rangle_\mathcal{D},
\end{equation}
which, using summation by parts (see Appendix \ref{app:summation}) can be written as (see Appendix \ref{app:potDerivNonCent})
\begin{equation}\label{eq:discROCPotBoundNonCent}
\underbrace{-c^2\langle(\mu_{x+}S)\dtd\dxp\Psiln, \dxp\Psiln\rangle_{\underline{\mathcal{D}}}}_{\let\scriptstyle\textstyle\delta_{t+}\mathfrak{v}} + \underbrace{c^2\Big( (\dtd\Psi_N^n)(\mup S_N)(\dxp\Psi_N^n) -(\dtd\Psi_0^n)(\mum S_0)(\dxm \Psi_0^n)\Big)}_{\mathfrak{b}},
\end{equation}
with boundary energy term $\mathfrak{b}$ as seen in Eq. \eqref{eq:energyFormBoundary} (for more detail, see Section \ref{sec:boundaries}).
Notice that $\mup S$ has been moved to the other side of the inner product according to match Eq. (9.14) in \cite{Bilbao2009}. We can also take $\mup S$ out of the inner product (as it is a coefficient here)
\begin{equation}
    -c^2(\mu_{x+}S)\langle\dtd\dxp\Psiln, \dxp\Psiln\rangle_{\underline{\mathcal{D}}},
\end{equation}
to use Eq. (2.22b) to end up with 
\begin{equation}
    \delta_{t+}\mathfrak{v} = \delta_{t+}\left(-\frac{c^2}{2}(\mu_{x+}S)\langle\dxp\Psiln, e_{t-}\dxp\Psiln\rangle_{\underline{\mathcal{D}}}\right),
\end{equation}
where (again, as $S$ acts as a coefficient) $\mup S$ can be inserted back into the inner product
\begin{equation}
   \mathfrak{v}=-\frac{c^2}{2}\langle(\mu_{x+}S)\dxp\Psiln, e_{t-}\dxp\Psiln\rangle_{\underline{\mathcal{D}}}.
\end{equation}
See Appendix \ref{app:proof} for a proof of this.

For the centered case, we need to use the weighted inner product as defined in \eqref{eq:weightedInnerProduct}:
\begin{equation}\label{eq:potContEnergyCent}
c^2\langle \dtd\Psiln, \dxm((\mup S)(\dxp\Psiln))\rangle_\mathcal{D}^{\epsilon_\text{l},\epsilon_\text{r}},
\end{equation}
which, after summation by parts becomes (see Appendix \ref{app:potDerivCent}) (\textit{Note that $\mathfrak{b}_\text{l}$ is subtracted})
\begin{equation}\label{eq:discROCPotBoundCent}
    \underbrace{-c^2\langle(\mu_{x+}S)\dtd\dxp\Psiln, \dxp\Psiln\rangle_{\underline{\mathcal{D}}}}_{\let\scriptstyle\textstyle\delta_{t+}\mathfrak{v}}\ +\ \mathfrak{b}_\text{r} - \mathfrak{b}_\text{l},
\end{equation}
where
\begin{subequations}
\begin{align}
 \mathfrak{b}_\text{r} &= c^2(\dtd\Psi_N^n)\left(\frac{\epsilon_\text{r}}{2}S_{N+1/2}(\dxp \Psi_N^n) + \left(1-\frac{\epsilon_\text{r}}{2}\right)S_{N-1/2}(\dxm\Psi_N^n)\right), \\
 \mathfrak{b}_\text{l} &= c^2(\dtd\Psi_0^n)\left(\frac{\epsilon_\text{l}}{2}S_{-1/2}(\dxm\Psi_0^n)+\left(1-\frac{\epsilon_\text{l}}{2}\right)S_{1/2}(\dxp \Psi_0^n))\right).
\end{align}
\end{subequations}
For the special cases of $\epsilon_\text{r} = S_{N-1/2}/\mu_{xx}S_N$ and $\epsilon_\text{l} = S_{1/2}/\mu_{xx}S_0$ the boundary terms become strictly dissipative (see Appendix \ref{app:potDerivCent} for the (painstaking) derivation of this)
\begin{subequations}\label{eq:centStrictDissip}
\begin{align}
    \mathfrak{b}_\text{r} &= c^2 (\dtd\Psi_N^n)S_{N-1/2}(2-\epsilon_\text{r})(\delta_{x\cdot}\Psi_N^n)\label{eq:centStrictDissipRight}\\
    \mathfrak{b}_\text{l} &= c^2 (\dtd\Psi_0^n)S_{1/2}(2-\epsilon_\text{l})(\delta_{x\cdot}\Psi_0^n)\label{eq:centStrictDissipLeft}
\end{align}
\end{subequations}

\subsubsection{Boundaries}\label{sec:boundaries}
Recalling the boundary term from Eq. \eqref{eq:discROCPotBoundNonCent} 
\begin{equation}
    \mathfrak{b} = \underbrace{ c^2(\dtd\Psi_N^n)(\mup S_N)(\dxp\Psi_N^n)}_{\mathfrak{b_\text{r}} \text{ (right boundary)}} -\underbrace{c^2(\dtd\Psi_0^n)(\mum S_0)(\dxm \Psi_0^n)}_{\mathfrak{b}_\text{l} \text{ (left boundary)}}
\end{equation}
We first consider the left boundary. Recalling Eq. \eqref{eq:centNonCentBound}, we see that for the non-centered case, $\Psi_{-1}^n = \Psi_0^n$. This means that $(\dxm\Psi_0^n) = 0 $ yielding
\begin{equation}
    \mathfrak{b}_\text{l} = c^2(\dtd\Psi_0^n)(S_{-1/2})(0) = 0.
\end{equation}
For the centered case, $\Psi_{-1}^n = \Psi_1^n$. Substituting this into the equation for the left boundary yields
\begin{equation}
    \mathfrak{b}_\text{l} = \frac{c^2}{2kh}(\Psi_0^{n+1} - \Psi_0^{n-1})(S_{-1/2})(\Psi_0^n - \Psi_1^n)
\end{equation}
Continuing with the right boundary it can be seen that the lossless case, where $(\dtd \Psi_N^n) = 0$ results in 
\begin{equation}
    \mathfrak{b}_\text{r} = c^2(0)(S_{N+1/2})(\dxp\Psi_N^n) = 0.
\end{equation}
The damped condition, however, is much more interesting. In the non-centered case, we can substitute this into the definition found in Eq. \eqref{eq:nonCentRadBound} to get
\begin{equation}
    \mathfrak{b}_\text{r} = c^2(\dtd\Psi_N^n)( S_{N+1/2})(-a_1\dtd\Psi_N^n - a_2\mu_{t\cdot}\Psi_N^n),
    \end{equation}
    and using the identity
    \begin{equation}
    (\dtd u)(\mu_{t\cdot}u) = \frac{1}{2} \dtd(u)^2,
\end{equation}
we end up with
\begin{equation}
    \mathfrak{b}_\text{r} = c^2 S_{N+1/2}\left(-a_1(\dtd\Psi_N^n)^2 - \frac{a_2}{2}\dtd(\Psi_N^n)^2\right).
\end{equation}
As this boundary both has damping with energy term $\mathfrak{q}_\text{b}$ and inertia with stored energy $\mathfrak{h}_\text{b}$ it can be written in the form found in Eq. \eqref{eq:radiatingEnergyForm}. First, substituting $\mathfrak{b}_\text{r}$ into Eq. \eqref{eq:energyFormBoundary} yields %we can subdivide $\mathfrak{b}_r$ as follows:
\begin{align}
\delta_{t+}\mathfrak{h} &= c^2 S_{N+1/2}\left(-a_1(\dtd\Psi_N^n)^2 - \frac{a_2}{2}\dtd(\Psi_N^n)^2\right)\\
\delta_{t+}\mathfrak{h} + \frac{c^2S_{N+1/2}a_2}{2}\delta_{t+}\mu_{t-} (\Psi_N^n)^2 &= -c^2S_{N+1/2}a_1(\dtd\Psi_N^n)^2
\end{align}
%
\begin{equation}
    \begin{gathered}
    \delta_{t+}\left(\mathfrak{h} + \mathfrak{h}_\text{b}\right) = -\mathfrak{q}_\text{b} \quad \text{where}\\
    \mathfrak{h}_\text{b} = \frac{c^2S_{N+1/2}a_2}{2}\mu_{t-} (\Psi_N^n)^2 \quad \text{and} \quad
    \mathfrak{q}_\text{b} = c^2S_{N+1/2}a_1(\dtd\Psi_N^n)^2,
    \end{gathered}
\end{equation}

which is exactly what we find in \cite{Bilbao2009} underneath Eq. (9.15).

For the centered case, recalling Eqs. \eqref{eq:centRadBound} and \eqref{eq:centStrictDissipRight} we can follow similar steps to end up at
\begin{equation}
    \mathfrak{b}_\text{r} = c^2S_{N-1/2}(2-\epsilon_\text{r})(-a_1(\dtd\Psi_N^n)^2-\frac{a_2}{2}\dtd(\Psi_N^n)^2).
\end{equation}
When written in the form found in Eq. \eqref{eq:radiatingEnergyForm} we obtain
\begin{equation}
    \begin{gathered}
    \delta_{t+}\left(\mathfrak{h} + \mathfrak{h}_\text{b}\right) = -\mathfrak{q}_\text{b} \quad \text{where}\\
    \mathfrak{h}_\text{b} = \frac{c^2S_{N-1/2}(2-\epsilon_\text{r})a_2}{2}\mu_{t-} (\Psi_N^n)^2 \quad \text{and} \quad
    \mathfrak{q}_\text{b} = c^2S_{N-1/2}(2-\epsilon_\text{r})a_1(\dtd\Psi_N^n)^2,
    \end{gathered}
\end{equation}
% The centered case found in Eq. \eqref{eq:centRadBound} needs to be rewritten in the form $\dxp\Psi_N^n = \hdots$ to be able to be substituted into the right boundary term. This is done as follows
% \begin{align}
%     \delta_{x\cdot}\Psi_N^n &= -a_1\dtd\Psi_N^n - a_2\mu_{t\cdot}\Psi_N^n\nonumber\\
%     \Psi_{N+1}^n-\Psi_{N-1}^n &= 2h\left(-a_1\dtd\Psi_N^n - a_2\mu_{t\cdot}\Psi_N^n\right)\nonumber\\
%     \Psi_{N+1}^n-\Psi_N^n+\Psi_N^n-\Psi_{N-1}^n &= 2h\left(-a_1\dtd\Psi_N^n - a_2\mu_{t\cdot}\Psi_N^n\right)\nonumber\\
%     \dxp\Psi_N^n+\dxm\Psi_N^n &= 2\left(-a_1\dtd\Psi_N^n - a_2\mu_{t\cdot}\Psi_N^n\right)\nonumber\\
%     \dxp\Psi_N^n &= 2\left(-a_1\dtd\Psi_N^n - a_2\mu_{t\cdot}\Psi_N^n\right)-\dxm\Psi_N^n
% \end{align}
% which can then be substituted into the right boundary term
% \begin{align}
%     \mathfrak{b}_\text{r} &=  c^2(\dtd\Psi_N^n)(S_{N+1/2})(2\left(-a_1\dtd\Psi_N^n - a_2\mu_{t\cdot}\Psi_N^n\right)-\dxm\Psi_N^n)\nonumber\\
% &= c^2S_{N+1/2}\left(-2a_1(\dtd\Psi_N^n)^2 - a_2\dtd(\Psi_N^n)^2-(\dtd\Psi_N^n)(\dxm\Psi_N^n)\right)
% \end{align}
% \SWcomment[Here, we can write it in the same form as Eq. \eqref{eq:radiatingEnergyForm} with
% \begin{equation}
%     \mathfrak{h}_\text{b} = c^2S_{N+1/2}a_2\mu_{t-} (\Psi_N^n)^2  \quad \text{and} \quad
% \mathfrak{q}_\text{b} = 2c^2S_{N+1/2}a_1(\dtd\Psi_N^n)^2,
% \end{equation}
% with a \textbf{leftover term}
% \begin{equation}
%     \mathfrak{?}_\text{b} = -(\dtd\Psi_N^n)(\dxm\Psi_N^n).
% \end{equation}]

\section{Time-varying System}
The PDE for the time-varying version of Webster's equation is (Eq. (9.23) in \cite{Bilbao2009})
\begin{equation}
     \partial_t(S\partial_t\Psi) = c^2 \partial_x(S\partial_x\Psi)
\end{equation}
which can be discretised as (right below Eq. (9.23) in \cite{Bilbao2009})
\begin{equation}
    \delta_{t+}\big((\mu_{t-}\bar S_l^n)(\delta_{t-}\Psiln)\big) = c^2\dxp\big((\muTT \Sm^n)(\dxm\Psiln)\big)
\end{equation}
The update scheme for this time varying system can be written as
\begin{equation}
    \begin{aligned}
        (\mu_{t+}\bar S_l^n)\Psinp &= \Big(2 \mu_{tt}\bar S_l^n -  \lambda^2(\muTT \Sp^n + \muTT \Sm^n)\Big) \Psiln \\
        &+ \lambda^2 \Big((\muTT \Sp^n)\Psilp + (\muTT\Sm)\Psilm\Big) - (\mu_{t-}\bar S_l^n)\Psinm
    \end{aligned}
\end{equation}

\section{Excitation}
To excite the system, an input impedance can be defined as follows
\begin{equation}
\delta_{x\cdot}\Psi_0^n = -v_\text{in}^n.
\end{equation} 
Rewriting to
\begin{equation}
    \frac{1}{2h}(\Psi_1^n-\Psi_{-1}^n) = -v_\text{in}^n \quad \Rightarrow \quad \Psi_{-1}^n = \Psi_1^n + 2hv_\text{in}^n
\end{equation}
and substituting this in Eq. \eqref{eq:webstersUpdateEq} at $l=0$ yields
\begin{align}
    \Psi_0^{n+1}&= 2(1-\lambda^2)\Psi_0^n-\Psi_0^{n-1}+ \frac{\lambda^2S_{1/2}}{\bar S_0}\Psi_1^n + \frac{\lambda^2S_{-1/2}}{\bar S_0}\Psi_1 + \frac{2h\lambda^2S_{-1/2}}{\bar S_0}v_\text{in}^n,\nonumber\\
    \Psi_0^{n+1}&= 2(1-\lambda^2)\Psi_0^n-\Psi_0^{n-1}+ \frac{2\lambda^2(S_{1/2}+S_{-1/2})}{S_{1/2}+S_{-1/2}}\Psi_1^n+ \frac{2h\lambda^2S_{-1/2}}{\bar S_0}v_\text{in}^n,\nonumber\\
    \Psi_0^{n+1}&= 2(1-\lambda^2)\Psi_0^n-\Psi_0^{n-1}+ 2\lambda^2\Psi_1^n+ \frac{2h\lambda^2S_{-1/2}}{\bar S_0}v_\text{in}^n,
\end{align}
which is identical to Eq. (19.58) in \cite{Bilbao2018}.
\section{Damping}
As said in \cite{Bilbao2013}, a fractional derivative can be used for viscothermal damping. This, in discrete time can be approximated using the bilinear transform (or Tustin's transformation).

Following \cite{Chen2002}:
\begin{equation}
    (\omega(\z))^r = \left(\frac{2}{k}\right)^r \left(\frac{1-\z}{1+\z}\right)^r.
\end{equation}
Then, if we consider $r = 3$:
\begin{equation}
    \begin{aligned}
    (\omega(\z))^3 &= \left(\frac{2}{k}\right)^3\left(\frac{(1-\z)^3}{(1+\z)^3}\right)\\
    &= \left(\frac{2}{k}\right)^3\left(\frac{1-\z-2\z+2z^{-2}+z^{-2}-z^{-3}}{1+\z+2\z+2z^{-2}+z^{-2}+z^{-3}}\right)\\
    &= \left(\frac{2}{k}\right)^3\left(\frac{1-3\z+3z^{-2}-z^{-3}}{1+3\z+3z^{-2}+z^{-3}}\right)
    \end{aligned}
\end{equation}
We can do the same if we follow Eq. (2) from \cite{Chen2002} with $n=3$:
\begin{equation}
    \begin{aligned}
    (\omega(\z))^3 &= \left(\frac{2}{k}\right)^3\frac{A_3(\z,3)}{A_3(\z,-3)}\\
    &= \left(\frac{2}{k}\right)^3\frac{-\frac{1}{3}3z^{-3}+\frac{1}{3}3^2z^{-2}-3\z+1}{-\frac{1}{3}(-3)z^{-3}+\frac{1}{3}(-3)^2z^{-2}-(-3)\z+1}\\
    &=\left(\frac{2}{k}\right)^3\frac{-z^{-3}+3z^{-2}-3\z+1}{z^{-3}+3z^{-2}+3\z+1},
    \end{aligned}
\end{equation}
which is equivalent to (19).

\subsection{Muir-recursion}
The Muir-recursion is described as
\begin{equation}\label{eq:muir}
    A_n(\z,r) = A_{n-1}(\z,r) -c_nz^nA_{n-1}(z,r) \quad \text{with} \quad A_0(z^{-1},r) = 1
\end{equation}
and
\begin{equation}
    c_n=
    \begin{cases}
    r/n & n \text{ is odd},\\
    0 & n \text{ is even.}
    \end{cases}
\end{equation}
If we let $n=3$ we get
\begin{equation}
    \begin{aligned}
        A_3(\z,r)&= A_2(\underbrace{(\z)^{-1}}_{z},r) - \frac{r}{3}(\z)^3A_2(\z,r)\\
        &= A_1((z)^{-1},r) - \frac{r}{3}z^{-3}A_1((\z)^{-1},r)\\
        &= A_0((\z)^{-1},r) - r(\z)^1A_0(\z,r)-\frac{r}{3}z^{-3}\left(A_0((z)^{-1},r)-rz^1A_0(z,r)\right)\\
        &(if\ A_0(z,r) = 1)\\
        &=1-r\z-\frac{r}{3}z^{-3}\left(1-rz\right)\\
        &=1-r\z+\frac{r^2}{3}z^{-2}-\frac{r}{3}z^{-3}
    \end{aligned}
\end{equation}
\subsection{Implementation of Muir-recursion}
Classes and member variables

\begin{itemize}
    \item \texttt{Z}
    \subitem \texttt{\type[double] coeff}
    \subitem \texttt{\type[int] power}
    \item \texttt{Equation\textless M\textgreater}
    \subitem \texttt{\type[std::vector\textless M\textgreater] zs} (of length M+1)
    \subitem \texttt{\type[int] zSign} (that only takes the values $-1$ and $1$).
\end{itemize}

There are a few observations that I made that helped the implementation
\begin{itemize}
    \item 
\end{itemize}

\subsubsection{The \texttt{swapCoeffs (\type[int] amount)} function}
This function is called when a \texttt{Z} is multiplied onto an \texttt{Equation}, i.e. the last term of Eq. \eqref{eq:muir} ($z^nA_{n-1}(z,r)$). I observed that in this term of the recursion the following relationship is true:
\begin{equation}\label{eq:relationship}
    \text{For any } z^p\cdot A(z,r), \quad  p = p_{A+} + 2\cdot\text{sgn}(p_{A+}),
\end{equation}
where $p_{A+}$ is the largest  power whether positive or negative present in $A$ (i.e. furthest away from 0). For example, the following could occur in the recursion:
\begin{equation}\label{eq:recursionExample}
    z^3 (A_1(z^{-1}, r)) = z^3(1 - rz^{-1}) = z^3 - rz^2.
\end{equation}
In this case $p_{A+} = -1$, and $p = 3$ (as can also be seen from the relationship in \eqref{eq:relationship}).
\\
\\
\noindent Now, the implementation of this allows us to swap coefficients of the \texttt{Z}'s in the z-vector around
\begin{equation}
\text{swap-around index} = |p| / 2,
\end{equation}
where $p$ is -- again -- the power of the $z$ multiplied onto the equation. As this power is always odd, the swap-around index is an integer-and-a-half. For the above example, the \texttt{Z} is $z^3$ and the \texttt{Equation} is $A_1(z^{-1},r) = 1-rz^{-1}$. The coefficients-vector of the \texttt{Equation}, \texttt{$\{$1, -r, 0, 0$\}$} (with negative powers) can now be flipped around index abs(3) / 2 = 1.5, so between index $1$ and $2$. If this is done and the signs of the powers of the \texttt{Z}'s in the vector are flipped, the coefficient-vector of the solution looks like \texttt{$\{$0, 0, -r, 1$\}$} (with positive powers), which is shown in the Eq. \eqref{eq:recursionExample}, i.e. $0z^0+0z^1-rz^2+1z^3$.
     
\bibliographystyle{plain}
\bibliography{bibliography}
\appendix
\def\Psinplp{\Psi_{l+1}^{n+1}}
\def\Psinmlp{\Psi_{l+1}^{n-1}}

\section{Derivations}
\subsection{Webster's Update Equation}\label{app:webstersUpdateEq}

\begin{align}
    \frac{\Sbar}{k^2}(\Psinp - 2\Psiln+\Psinm) &= c^2\left((\dxp \Sm)(\mup \dxm \Psiln) + (\mup \Sm)(\dxp \dxm \Psiln)\right)\nonumber\\
    \Psinp - 2\Psiln+\Psinm &= \frac{c^2k^2}{\Sbar}\bigg(\frac{1}{h}(\Sp - \Sm)\frac{1}{2h}\overbrace{(\Psilp -\Psilm)}^{\mup\dxm\Psiln = \delta_{x\cdot}\Psiln}\nonumber\\
    &\qquad\qquad+\frac{1}{2}(\Sp + \Sm)\frac{1}{h^2}(\Psilp-2\Psiln+\Psilm)\bigg)\nonumber\\
    \Psinp &= 2\Psiln-\Psinm + \overbrace{\frac{\lambda^2}{2\Sbar}}^{\lambda = \frac{ck}{h}}\Big(\Sp\Psilp - \Sp\Psilm - \Sm\Psilp + \Sm\Psilm \nonumber\\
    &+ \Sp\Psilp + \Sp\Psilm + \Sm\Psilp + \Sm\Psilm - 2 (\Sp + \Sm)\Psiln\Big)\nonumber\\
    \Psinp &= 2\Psiln-\Psinm+ \frac{\lambda^2}{2\Sbar}\Big(2\Sp\Psilp + 2\Sm\Psilm - 4\Sbar\Psiln\Big)\nonumber\\
    \Psinp &= 2\Psiln-\Psinm+ \frac{\lambda^2\Sp}{\Sbar}\Psilp + \frac{\lambda^2\Sm}{\Sbar}\Psilm - 2\lambda^2\Psiln\nonumber\\
    \Psinp &= 2(1-\lambda^2)\Psiln-\Psinm+ \frac{\lambda^2\Sp}{\Sbar}\Psilp + \frac{\lambda^2\Sm}{\Sbar}\Psilm,
\end{align}

\subsection{Centered Radiation}
\label{app:centeredRad} 
\begin{align}
    \Psi_N^{n+1} = &\ 2(1-\lambda^2)\Psi_N^n-\Psi_N^{n-1}+ \frac{\lambda^2S_{N+1/2}}{\bar S_N}\Psi_{N+1}^n + \frac{\lambda^2S_{N-1/2}}{\bar S_N}\Psi_{N-1}^n\nonumber\\
    \Psi_N^{n+1} = &\ 2(1-\lambda^2)\Psi_N^n-\Psi_N^{n-1}\nonumber\\
    &+\frac{\lambda^2S_{N+1/2}}{\bar S_N}\left[h\left(-\frac{a_1}{k}(\Psi_N^{n+1} - \Psi_N^{n-1}) - a_2(\Psi_N^{n+1} + \Psi_N^{n-1})\right)+\Psi_{N-1}^n\right]\nonumber\\
    &+ \frac{\lambda^2S_{N-1/2}}{\bar S_N}\Psi_{N-1}^n\nonumber\\
    \Psi_N^{n+1} = &\ 2(1-\lambda^2)\Psi_N^n-\Psi_N^{n-1}+ \frac{h\lambda^2S_{N+1/2}}{\bar S_N}\left[\left(-\frac{a_1}{k} - a_2\right)\Psi_N^{n+1} + \left(\frac{a_1}{k}-a_2\right)\Psi_N^{n-1}\right]\nonumber\\
    &+ \frac{\lambda^2(S_{N+1/2}+S_{N-1/2})}{\bar S_N}\Psi_{N-1}^n\nonumber\\
    \!\!\!\!\!\!\!\!\!\!\!\!\!\!\!\!\!\!\!\!\!\!\!\!\!\!\!\!\!\!\!\!\!\!\!\!\!\!\!\!\!\!\!\!\bigg(1+\Big(\frac{a_1}{k}+a_2\Big)\frac{h\lambda^2S_{N+1/2}}{\bar S_N}\bigg)\Psi_N^{n+1} =&\  2(1-\lambda^2)\Psi_N^n-\Psi_N^{n-1}+\frac{h\lambda^2S_{N+1/2}}{\bar S_N}\left(\frac{a_1}{k}-a_2\right)\Psi_N^{n-1} + 2\lambda^2\Psi_{N-1}^n\nonumber\\
    \Psi_N^{n+1} = &\  \frac{2(1-\lambda^2)\Psi_N^n-\Psi_N^{n-1}+\frac{h\lambda^2S_{N+1/2}}{\bar S_N}\left(\frac{a_1}{k}-a_2\right)\Psi_N^{n-1} + 2\lambda^2\Psi_{N-1}^n}{\left(1+\left(\frac{a_1}{k}+a_2\right)\frac{h\lambda^2S_{N+1/2}}{\bar S_N}\right)}
\end{align}
\subsection{Non-centered Radiation}\label{app:nonCentRad}
\begin{align}
    \Psi_N^{n+1} = &\ 2(1-\lambda^2)\Psi_N^n-\Psi_N^{n-1}+ \frac{\lambda^2S_{N+1/2}}{\bar S_N}\Psi_{N+1}^n + \frac{\lambda^2S_{N-1/2}}{\bar S_N}\Psi_{N-1}^n\nonumber\\
    \Psi_N^{n+1} = &\ 2(1-\lambda^2)\Psi_N^n-\Psi_N^{n-1}\nonumber\\
    &+\frac{\lambda^2S_{N+1/2}}{\bar S_N}\left[h\left(-\frac{a_1}{2k}(\Psi_N^{n+1} - \Psi_N^{n-1}) - \frac{a_2}{2}(\Psi_N^{n+1} + \Psi_N^{n-1})\right)+\Psi_N^n\right]\nonumber\\
    &+ \frac{\lambda^2S_{N-1/2}}{\bar S_N}\Psi_{N-1}^n\nonumber\\
    \Psi_N^{n+1} = &\ 2(1-\lambda^2)\Psi_N^n-\Psi_N^{n-1}+ \frac{h\lambda^2S_{N+1/2}}{\bar S_N}\left[\left(-\frac{a_1}{2k} - \frac{a_2}{2}\right)\Psi_N^{n+1} + \left(\frac{a_1}{k}-\frac{a_2}{2}\right)\Psi_N^{n-1}\right]\nonumber\\
    &+ \frac{\lambda^2S_{N+1/2}}{\bar S_N}\Psi_{N}^n+ \frac{\lambda^2S_{N-1/2}}{\bar S_N}\Psi_{N-1}^n \nonumber\\
    \!\!\!\!\!\!\!\!\!\!\!\!\!\!\!\!\!\!\!\!\!\!\!\!\!\!\!\!\!\!\!\!\!\!\!\!\!\!\!\!\!\!\!\!\bigg(1+\Big(\frac{a_1}{2k}+\frac{a_2}{2}\Big)\frac{h\lambda^2S_{N+1/2}}{\bar S_N}\bigg)\Psi_N^{n+1} =&\  2(1-\lambda^2)\Psi_N^n-\Psi_N^{n-1}+\frac{h\lambda^2S_{N+1/2}}{\bar S_N}\left(\frac{a_1}{2k}-\frac{a_2}{2}\right)\Psi_N^{n-1}\nonumber\\
    &+ \frac{\lambda^2S_{N+1/2}}{\bar S_N}\Psi_{N}^n+ \frac{\lambda^2S_{N-1/2}}{\bar S_N}\Psi_{N-1}^n \nonumber\\
    \Psi_N^{n+1} = &\  \frac{2(1-\lambda^2)\Psi_N^n-\Psi_N^{n-1}+\frac{h\lambda^2S_{N+1/2}}{\bar S_N}\left(\frac{a_1}{2k}-\frac{a_2}{2}\right)\Psi_N^{n-1} + \frac{\lambda^2S_{N+1/2}}{\bar S_N}\Psi_{N}^n + \frac{\lambda^2S_{N-1/2}}{\bar S_N}\Psi_{N-1}^n}{\left(1+\left(\frac{a_1}{2k}+\frac{a_2}{2}\right)\frac{h\lambda^2S_{N+1/2}}{\bar S_N}\right)}
\end{align}
\section{Inner Product with $S$}\label{app:proof}
\begin{equation}
\begin{aligned}
        &-c^2\langle\dtd\dxp\Psiln, \mup S\dxp\Psiln\rangle_\mathcal{D},\\
        \Longleftrightarrow \quad
        &-c^2\sum_\mathcal{D} h(\dtd\dxp\Psiln)( \mup S\dxp\Psiln)\\
        \Longleftrightarrow \quad & -c^2\sum_\mathcal{D}\frac{h}{4kh^2}\left(\Psinplp - \Psinmlp-\Psinp+\Psinm\right)\left(S_{l+1} + S_l\right)\left(\Psilp-\Psiln\right)\\
        \Longleftrightarrow \quad
        & -c^2\sum_\mathcal{D}\frac{h}{4kh^2}\big(\Psinplp\Psilp-\Psinplp\Psiln-\Psinp\Psilp+\Psinp\Psiln\\
        &\qquad\qquad\qquad\!\!\!\!\!\;-\Psilp\Psinmlp+\Psilp\Psinm+\Psiln\Psinmlp-\Psiln\Psinm\big)\left(S_{l+1}+S_l\right)\\
        \Longleftrightarrow \quad
        & -c^2\sum_\mathcal{D}\frac{h}{4h^2}\delta_{t+}\left(\Psilp\Psinmlp-\Psilp\Psinm-\Psiln\Psinmlp+\Psiln\Psinm\right)\left(S_{l+1}+S_l\right)\\
        \Longleftrightarrow \quad
        & \delta_{t+}\left(-c^2\sum_\mathcal{D}\frac{h}{2}(\dxp\Psiln)(\dxp\Psinm)(\mup S)\right)
        \\
        \Longleftrightarrow \quad
        & \delta_{t+}\left(-\frac{c^2}{2}\langle(\mup S)\dxp\Psiln,e_{t-}\Psiln\rangle_\mathcal{D}\right),
    \end{aligned}
\end{equation}
which (except for the use of $c$ instead of $\gamma$) is equivalent to Eq. (9.14).

\section{Summation by parts (proving equations (5.25) and (5.26))}\label{app:summation}
The general form of summation by parts of two functions $f$ and $g$, the latter of which has a single spatial derivative is as follows (Eq. 5.25) in \cite{Bilbao2009}):
\begin{equation}\label{eq:genSum}
    \langle f, \delta_{x+}g \rangle_\mathcal{D} = \sum_{l=d_-}^{d_+}hf_l\frac{1}{h}(g_{l+1}-g_l) = -\langle \delta_{x-}f, g\rangle_\mathcal{D} + f_{d_+}g_{d_++1} - f_{d_--1}g_{d_-}
\end{equation}
Proving this, we can use an example domain of $\mathcal{D}\in[0,...,3]$:
\begin{equation}
    \begin{aligned}
        \sum_{l=0}^3hf_l\frac{1}{h}(g_{l+1}-g_l) &= f_0g_1-f_0g_0 + f_1g_2-f_1g_1 + f_2g_3-f_2g_2 + f_3g_4-f_3g_3\\
        &= g_0(f_{-1}-f_0) + g_1(f_0-f_1) + g_2(f_1-f_2) + g_3(f_2-f_3) + f_3g_4 - f_{-1}g_0 \\
        &= -g_0(f_0-f_{-1}) - g_1(f_1-f_0) - g_2(f_2-f_1) - g_3(f_3-f_2) + f_3g_4 - f_{-1}g_0\\
        &= -\sum_{l=0}^3 g_l(f_l-f_{l-1}) + f_3g_4 - f_{-1}g_0\\
        &= -\sum_{l=0}^3 h\delta_{x-}f_lg_l + f_3g_4 - f_{-1}g_0
    \end{aligned}
\end{equation}
Then replacing the applied domain with the general form yields
\begin{equation}
    -\sum_{l=d_-}^{d_+}h\delta_{x-}f_lg_l + f_{d_+}g_{d_++1} - f_{d_--1}g_{d_-} \quad \Rightarrow \quad -\langle \delta_{x-}f, g\rangle_\mathcal{D} + f_{d_+}g_{d_++1} - f_{d_--1}g_{d_-}
\end{equation}
which is the general form in \eqref{eq:genSum}.

In the same way, we can prove Eq. (5.26):
\begin{equation}
    \begin{aligned}
    \langle f,\delta_{x+}g\rangle_{\underline{\mathcal{D}}}=\sum_{l=0}^2hf_l\delta_{x+}g_l &= f_0g_1-f_0g_0 + f_1g_2-f_1g_1 + f_2g_3-f_2g_2 \\
    &= - g_1(f_1-f_0) - g_2(f_2-f_1) - g_3(f_3-f_2) + f_3g_3 - f_0g_0
    \\
    &= -\sum_{l=1}^3h\delta_{x-}f_lg_l + f_3g_3 - f_0g_0
    \end{aligned}
\end{equation}
Again, going general yields
\begin{equation}
    -\sum_{l=d_-+1}^{d_+}h\delta_{x-}f_lg_l + f_{d_+}g_{d_+}-f_{d_-}g_{d_-} \quad \Rightarrow \quad -\langle \delta_{x-}f,g\rangle_{\overline{\mathcal{D}}} + f_{d_+}g_{d_+}-f_{d_-}g_{d_-}
\end{equation}
\subsection{Proving (5.27)}
Using again an example domain of $\mathcal{D}\in [0,\hdots,3]$ 
\begin{equation}
    \begin{aligned}
        \langle f,\delta_{xx}g \rangle_\mathcal{D} &= \sum_{l=0}^3 h f_l\frac{1}{h^2}(g_{l+1}-2g_l+g_{l-1})\\
        &= \frac{1}{h}\Big(f_0g_{-1} - 2 f_0g_0 + f_0g_1 + f_1g_0 - 2 f_1g_1 + f_1g_2+ f_2g_{1} - 2 f_2g_2 + f_2g_3 + f_3g_2 - 2f_3g_3 + f_3g_4 \Big)\\
        &=\frac{1}{h}\Big(f_0g_{-1} + g_0(f_{-1}-2f_0+f_1) - g_0f_{-1} + g_1(f_0-2f_1+f_2)+\\
        &\qquad\ \ \;g_2(f_1-2f_2+f_3)+ g_3(f_2-2f_3+f_4) - g_3f_4 + f_3g_4\Big)\\
        &=\frac{1}{h}\Big(\sum_{l=0}^3 h^2 g_l\delta_{xx}f_l + f_0g_{-1}-g_0f_{-1} - g_3f_4 + f_3g_4\Big)\\
        &= \sum_{l=0}^3 h g_l\delta_{xx}f_l + \frac{1}{h}\Big(-f_0g_0+f_0g_{-1}+f_0g_0-f_{-1}g_0+f_3g_4-f_3g_3-f_4g_3+f_3g_3\Big)\\
        &= \langle \delta_{xx}f, g \rangle_{\mathcal{D}} -f_0\delta_{x-}g_0+g_0\delta_{x-}f_0+f_3\delta_{x+}g_3-g_3\delta_{x+}f_3    
    \end{aligned}
\end{equation}
\subsection{The same with a primed inner product}
The primed inner product from \cite{Bilbao2009} is defined as
\begin{equation}
    \langle f^n,g^n \rangle_\mathcal{D}' = \sum_{d_-+1}^{d_+-1}hf_l^ng_l^n+\frac{h}{2}f_{d-}^ng_{d-}^n+\frac{h}{2}f_{d+}^ng_{d+}^n,
\end{equation}
so essentially a regular inner product with the outer terms scaled by $1/2$.

Using the same case as above we get
\begin{equation}
\begin{aligned}
    \langle f,\delta_{xx}g \rangle_\mathcal{D}' &= \sum_{l=1}^2 h f_l\frac{1}{h^2}(g_{l+1}-2g_l+g_{l-1}) + \frac{h}{2}f_0\frac{1}{h^2}(g_1-2g_0+g_{-1}) + f_3 \frac{1}{h^2}(g_{4}-2g_3+g_2)\\
    &=\frac{1}{h}\Big(\frac{1}{2}(f_0g_1-2f_0g_0+f_0g_{-1})+f_1g_2-2f_1g_1+f_1g_0\\
    &\qquad\ +f_2g_3-2f_2g_2+f_2g_1+\frac{1}{2}(f_3g_4-2f_3g_3+f_3g_2)\Big)\\
    & = \frac{1}{h}\Big(g_1(f_2-2f_1+f_0)-\frac{1}{2}f_0g_1+g_2(f_3-2f_2+f_1)-\frac{1}{2}f_3g_2\\
    &\qquad\ -f_0g_0+\frac{1}{2}f_0g_{-1}-f_3g_3+\frac{1}{2}f_3g_4+f_1g_0+f_2g_3)\Big)\\
    &= \frac{1}{h}\left(\sum_{l=1}^2h^2g_l\delta_{xx}f_l+\frac{1}{2}f_0g_{-1}-f_0g_0-\frac{1}{2}f_0g_1+f_1g_0+f_2g_3-\frac{1}{2}f_3g_2-f_3g_3+\frac{1}{2}f_3g_4\right)\\
    &= \langle \delta_{xx}f,g\rangle_\mathcal{D}' -\frac{h}{2}g_0\delta_{xx}f_0-\frac{h}{2}g_3\delta_{xx}f_3\\
    &\qquad\ +\frac{1}{h}\left(\frac{1}{2}f_0g_{-1}-f_0g_0-\frac{1}{2}f_0g_1+f_1g_0+f_2g_3-\frac{1}{2}f_3g_2-f_3g_3+\frac{1}{2}f_3g_4\right)\\
    &= \langle \delta_{xx}f,g\rangle_\mathcal{D}' -\frac{h}{2}g_0\frac{1}{h^2}(f_1-2f_0+f_{-1})-\frac{h}{2}g_3\frac{1}{h^2}(f_4-2f_3+f_2)\\
    &\qquad\ +\frac{1}{h}\left(\frac{1}{2}f_0g_{-1}-f_0g_0-\frac{1}{2}f_0g_1+f_1g_0+f_2g_3-\frac{1}{2}f_3g_2-f_3g_3+\frac{1}{2}f_3g_4\right)\\
    &= \langle \delta_{xx}f,g\rangle_\mathcal{D}'+\frac{1}{h}\Big( -\frac{1}{2}f_1g_0+f_0g_0-\frac{1}{2}f_{-1}g_0-\frac{1}{2}f_4g_3+f_3g_3-\frac{1}{2}f_2g_3\\
    &\qquad\ +\frac{1}{2}f_0g_{-1}-f_0g_0-\frac{1}{2}f_0g_1+f_1g_0+f_2g_3-\frac{1}{2}f_3g_2-f_3g_3+\frac{1}{2}f_3g_4\Big)\\
    &= \langle \delta_{xx}f,g\rangle_\mathcal{D}'+\frac{1}{2h}\Big(f_1g_0-f_{-1}g_0-f_4g_3+f_2g_3-f_0g_1+f_0g_{-1}-f_3g_2+f_3g_4\Big)\\
    &= \langle \delta_{xx}f,g\rangle_\mathcal{D}'+\frac{1}{2h}(f_1-f_{-1})g_0 - \frac{1}{2h}(g_1-g_{-1})f_0+\frac{1}{2h}(g_4-g_2)f_3-\frac{1}{2h}(f_4-f_2)g_3\\
    &= \langle \delta_{xx}f,g\rangle_\mathcal{D}'+ g_0\delta_{x\cdot}f_0 - f_0\delta_{x\cdot}g_0-g_3\delta_{x\cdot}f_3+f_3\delta_{x\cdot}g_3
\end{aligned}
\end{equation}
which is identical to the identity presented in Problem 5.8 in \cite{Bilbao2009}. Note that the difference with the regular inner product is that the boundary terms contain a centered derivative rather than a non-centered one. 

\section{Potential energy derivation}\label{app:potDeriv}
\subsection{...for the non-centered case}\label{app:potDerivNonCent}
Recalling Eq. \eqref{eq:potContEnergy} and disregarding the multiplication with $c^2$ for now, we set $f=\dtd\Psi$ and $g = (\mup S)(\dxp \Psi)$ and domain $\mathcal{D} \in [0,N]$ which yields
\begin{align}
    \langle f, \dxm g\rangle_\mathcal{D} = \sum_{l=0}^N hf_l\dxm g_l =&\  f_0(g_0-g_{-1}) + f_1(g_1-g_0) + \hdots f_N(g_N-g_{N-1}) \nonumber\\
    =&\ - g_0(f_1-f_0) - g_1(f_2-f_1) - \hdots g_{N-1}(f_N-f_{N-1})  + f_Ng_N - f_0g_{-1}\nonumber\\
    =&\  -\sum_{l=0}^{N-1}h\dxp f_lg_l + f_Ng_N - f_0g_{-1}\nonumber\\
    =&\  -\langle\dxp f,  g\rangle_\mathcal{\underline{D}} + f_Ng_N - f_0g_{-1}\nonumber\\
    &\nonumber \\
    = &\ -\langle\dxp\dtd\Psi,(\mup S) \dxp\Psi\rangle_{\underline{\mathcal{D}}} + (\dtd\Psi_N)(\mup S_N)(\dxp\Psi_N) \nonumber\\
    &\ \qquad\qquad\qquad- (\dtd\Psi_0)\underbrace{(\mup S_{-1})}_{(\mum S_0)}\underbrace{(\dxp\Psi_{-1})}_{(\dxm \Psi_0)}
\end{align} 

\subsection{..using the primed inner product}
Again, using domain $\mathcal{D} \in [0, \hdots, N]$ and $N = 3$, we get
\begin{equation}
\begin{aligned}
    \langle f, \dxm g\rangle_\mathcal{D}' &= \sum_{l=1}^{N-1} hf_l\dxm g_l + \frac{h}{2}f_0\dxm g_0 + \frac{h}{2}f_3\dxm g_3\\
    &= \frac{h}{2}f_0\frac{1}{h}(g_0-g_{-1})+hf_1\frac{1}{h}(g_1-g_0)+ hf_2\frac{1}{h}(g_2-g_1)+ \frac{h}{2}f_3\frac{1}{h}(g_3-g_{2})\\
    &= \frac{1}{2}f_0g_0-\frac{1}{2}f_0g_{-1}+f_1g_1-f_1g_0+f_2g_2-f_2g_1+\frac{1}{2}f_3g_3-\frac{1}{2}f_3g_2\\
    &= -g_0(f_1-f_0) - \frac{1}{2}f_0g_0-g_1(f_2-f_1)-g_2(f_3-f_2) + \frac{1}{2}f_3g_2-\frac{1}{2}f_0g_{-1}+\frac{1}{2}f_3g_3\\
    &= -\sum_{l=0}^{N-1}hg_l\dxp f_l - \frac{1}{2}f_0g_0-\frac{1}{2}f_0g_{-1}+\frac{1}{2}f_3g_3+\frac{1}{2}f_3g_2\\
    &= -\langle \dxp f, g \rangle_{\underline{\mathcal{D}}}-f_0(\mum g_0)+f_3(\mum g_3)
\end{aligned}
    \end{equation}
Then, filling in the $f=\dtd\Psi$ and $g = (\mup S)(\dxp \Psi)$ yields
\begin{equation}
    \begin{aligned}
        &-\langle \dxp \dtd\Psi, (\mup S)(\dxp \Psi)\rangle_\mathcal{\underline{D}} - (\dtd\Psi_0^n)\mum\big((\mup S_0)(\dxp \Psi_0^n)\big) + (\dtd\Psi_N^n)\mum\big((\mup S_N)(\dxp \Psi_N^n)\big)\\
        %\Longleftrightarrow \quad &-\langle \dxp \dtd\Psi, (\mup S)(\dxp \Psi)\rangle_\mathcal{\underline{D}}'- (\dtd\Psi_0^n)(\mu_{xx} S_0)(\delta_{x\cdot} \Psi_0^n) + (\dtd\Psi_N^n)(\mu_{xx} S_N)(\delta_{x\cdot} \Psi_N^n)   
    \end{aligned}
\end{equation}
which is close (!) but unfortunately doesn't give us the right answer. 

\subsection{(scratch that) ...using the more general weighed inner product! AKA solving Problem 9.5 AKA ... for the centered case}\label{app:potDerivCent}
Recalling Eq. \eqref{eq:potContEnergy}, but now applying the weighted inner product found in Eq. \eqref{eq:weightedInnerProduct} (with $f=\dtd\Psi$ and $g = S_{l+1/2}(\dxp\Psi)$) yields
\begin{equation}
    \begin{aligned}
        \langle f,\dxm g \rangle_{\mathcal{D}}^{\epsilon_\text{l},\epsilon_\text{r}} &= \sum_{l=1}^{N-1} h f_l\frac{1}{h}(g_l-g_{l-1}) + \frac{\epsilon_\text{l}}{2}hf_0\frac{1}{h}(g_0 - g_{-1})+ \frac{\epsilon_\text{r}}{2}hf_N\frac{1}{h}(g_N - g_{N-1})\\
        &=f_1g_1-f_1g_0+f_2g_2-f_2g_1+\hdots+f_{N-1}g_{N-1}-f_{N-1}g_{N-2}\\
        &\qquad+ \frac{\epsilon_\text{l}}{2}f_0(g_0 - g_{-1}) + \frac{\epsilon_\text{r}}{2}f_N(g_N-g_{N-1})\\
        &= -g_0(f_1-f_0) - f_0g_0 - g_1(f_2-f_1) - \hdots - g_{N-1}(f_N - f_{N-1}) + f_Ng_{N-1}\\
        &\qquad + \frac{\epsilon_\text{l}}{2}f_0(g_0 - g_{-1}) + \frac{\epsilon_\text{r}}{2}f_N(g_N-g_{N-1})\\
        &= -\langle \dxp f, g\rangle_{\underline{\mathcal{D}}} - f_0g_0 + \frac{\epsilon_\text{l}}{2}f_0(g_0 - g_{-1}) + f_Ng_{N-1} + \frac{\epsilon_\text{r}}{2}f_N(g_N-g_{N-1}).
    \end{aligned}
\end{equation}
Then, only looking at the boundaries, filling in the definitions for $f$ and $g$ and recalling the multiplication with $c^2$ yields
\begin{equation}
    \begin{aligned}
    \mathfrak{b}  &=-c^2(\dtd\Psi_0^n)\Big(S_{1/2}(\dxp\Psi_0^n)\Big) +\frac{\epsilon_\text{l}}{2}(\dtd\Psi_0^n)\Big(S_{1/2}(\dxp\Psi_0^n)-S_{-1/2}\overbrace{(\dxp\Psi_{-1}^n)}^{(\dxm\Psi_0^n)}\Big) \\
    & \qquad +c^2 (\dtd\Psi_N^n)\Big(S_{N-1/2}\underbrace{(\dxp\Psi_{N-1}^n)}_{(\dxm\Psi_{N}^n)}\Big)+  \frac{\epsilon_\text{r}}{2}(\dtd\Psi_{N}^n)\Big(S_{N+1/2}(\dxp\Psi_N^n) - S_{N-1/2}\underbrace{(\dxp \Psi_{N-1}^n)}_{(\dxm\Psi_{N}^n)}\Big)\\
    &= c^2(\dtd\Psi_N^n)\left(\frac{\epsilon_\text{r}}{2}S_{N+1/2}(\dxp \Psi_N^n) + \left(1-\frac{\epsilon_\text{r}}{2}\right)S_{N-1/2}(\dxm\Psi_N^n)\right)\\
    &\qquad - c^2(\dtd\Psi_0^n)\left(\frac{\epsilon_\text{l}}{2}S_{-1/2}(\dxm\Psi_0^n)+\left(1-\frac{\epsilon_\text{l}}{2}\right)S_{1/2}(\dxp \Psi_0^n))\right)
    \end{aligned}
\end{equation}
which is what is shown in Problem 9.5. 

Then, for the the centered radiating boundary condition shown in Eq. (9.16) in \cite{Bilbao2009}  %($\delta_{x\cdot}\Psi_N^n = -a_1\dtd\Psi_N^n - a_2\mu_{t\cdot}\Psi_N^n$) 
to be strictly dissipative we need to make the special choice for $\epsilon_\text{r} = S_{N-1/2} / \mu_{xx}S_N$. Only considering the right boundary and continuing with this choice of $\epsilon_\text{r}$ we get
\begin{equation}
    \begin{aligned}
        \mathfrak{b}_\text{r} &= c^2 (\dtd\Psi_N^n)\left(\frac{S_{N-1/2}}{2\mu_{xx}S_N}S_{N+1/2}(\dxp\Psi_N^n)+\left(1-\frac{S_{N-1/2}}{2\mu_{xx}S_N}\right)S_{N-1/2}(\dxm\Psi_N^n)\right)\\
        &= c^2 (\dtd\Psi_N^n)S_{N-1/2}\left(\frac{S_{N+1/2}}{2\mu_{xx}S_N}(\dxp\Psi_N^n)+\left(1-\frac{S_{N-1/2}}{2\mu_{xx}S_N}\right)(\dxm\Psi_N^n)\right)\\
        &=c^2 (\dtd\Psi_N^n)S_{N-1/2}\left(1-\frac{S_{N-1/2}}{2\mu_{xx}S_N}\right)\left(\frac{\frac{S_{N+1/2}(\dxp\Psi_N^n)}{2\mu_{xx}S_N}}{\left(1-\frac{S_{N-1/2}}{2\mu_{xx}S_N}\right)}+\dxm\Psi_N^n\right)\\
        &=c^2 (\dtd\Psi_N^n)S_{N-1/2}\left(1-\frac{\epsilon_\text{r}}{2}\right)\left(\frac{\frac{S_{N+1/2}(\dxp\Psi_N^n)}{2\mu_{xx}S_N}}{\left(\frac{2\mu_{xx} - S_{N-1/2}}{2\mu_{xx}S_N}\right)}+\dxm\Psi_N^n\right)\\
        &=c^2 (\dtd\Psi_N^n)S_{N-1/2}\left(1-\frac{\epsilon_\text{r}}{2}\right)\left(\frac{S_{N+1/2}(\dxp\Psi_N^n)}{2\mu_{xx}S_N - S_{N-1/2}}+\dxm\Psi_N^n\right)\\
        &=c^2 (\dtd\Psi_N^n)S_{N-1/2}\left(1-\frac{\epsilon_\text{r}}{2}\right)\left(\frac{S_{N+1/2}(\dxp\Psi_N^n)}{S_{N+1/2} + S_{N-1/2} - S_{N-1/2}}+\dxm\Psi_N^n\right)\\
        &=c^2 (\dtd\Psi_N^n)S_{N-1/2}\left(1-\frac{\epsilon_\text{r}}{2}\right)\left(\dxp\Psi_N^n+\dxm\Psi_N^n\right)\\
        &=c^2 (\dtd\Psi_N^n)S_{N-1/2}\left(1-\frac{\epsilon_\text{r}}{2}\right)\left(\frac{1}{h}\left(\Psi_{N+1}^n - \Psi_N^n + \Psi_N^n - \Psi_{N-1}^n\right)\right)\\
&= c^2 (\dtd\Psi_N^n)S_{N-1/2}\left(1-\frac{\epsilon_\text{r}}{2}\right)2(\delta_{x\cdot}\Psi_N^n)\\
&= c^2 (\dtd\Psi_N^n)S_{N-1/2}\left(2-\epsilon_\text{r}\right)(\delta_{x\cdot}\Psi_N^n)
    \end{aligned}
\end{equation}
The same can be done for $\mathfrak{b}_\text{l}$ with $\epsilon_\text{l} = S_{1/2}/\mu_{xx}S_0$ to get
\begin{equation}
    \mathfrak{b}_\text{l} = c^2(\dtd\Psi_0^n)S_{1/2}\left(2-\epsilon_\text{l}\right)(\delta_{x\cdot}\Psi_0^n)
\end{equation}
\end{document}
