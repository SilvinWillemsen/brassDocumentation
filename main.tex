\documentclass{article}
\usepackage[utf8]{inputenc}
\usepackage[left=3cm, right=3cm, top=2cm]{geometry}
\title{Brass}
\author{Silvin Willemsen}
\date{March 2020}

\usepackage{natbib}
\usepackage{graphicx}
\usepackage{appendix}
\usepackage{amsmath}
\usepackage{amsfonts}
\usepackage{amssymb}

\usepackage{xcolor}
\def\SBcomment[#1]{\textcolor{red}{#1}}
\def\SWcomment[#1]{\textcolor{blue}{#1}}
\def\SScomment[#1]{\textcolor{green}{#1}}

\def\dxp{\delta_{x+}}
\def\dxm{\delta_{x-}}
\def\mup{\mu_{x+}}
\def\mum{\mu_{x-}}
\def\Sp{S_{l+1/2}}
\def\Sm{S_{l-1/2}}
\def\Psilp{\Psi_{l+1}^n}
\def\Psilm{\Psi_{l-1}^n}
\def\Psinp{\Psi_l^{n+1}}
\def\Psinm{\Psi_l^{n-1}}
\def\Psiln{\Psi_l^n}
\def\Sbar{\bar S_l}

\begin{document}

\maketitle

\section{Introduction}
This document shows the work done and documentation on the brass part of the PhD.

Main references are \cite{Bilbao2009} and \cite{Bilbao2018}.

\section{Variable cross-section}
\subsection{Webster's equation}
The first main difference between the 1D brass PDE and the 1D wave equation is the possibility of having a variable cross-section. Following Section 19.3 from \cite{Bilbao2018}, the PDE for a 1D (axially symmetric) acoustic tube with variable cross-section is (also known as \textit{Webster}'s equation)
\begin{equation}\label{eq:webstersPDE}
    S\partial_t^2\Psi = c^2\partial_xS(\partial_x\Psi),
\end{equation}
with \textit{acoustic potential} $\Psi = \Psi(x,t)$ (m$^2$/s), $S = S(x)$ is the cross sectional area (m$^2$) and wave speed $c$ (m/s).

\subsection{Discretisation}
Introducing interleaved gridpoints at $n-1/2$ and $n+1/2$ for $S$, a we can discretise Eq. \eqref{eq:webstersPDE} (following \cite{Bilbao2018}) to
\begin{equation}\label{eq:discWebster}
    \Sbar \delta_{tt}\Psi^n_l = c^2\dxp(\Sm(\delta_{x-}\Psiln)),
\end{equation}
where
\begin{equation}
    \Sbar = \mu_{t+}\Sm = \frac{\Sp + \Sm}{2}.
\end{equation}
The right side of the equation in \eqref{eq:discWebster} contains an operator applied to two grid functions ($S$ and $\Psi$) multiplied onto each other. In order to expand this, we need to use the product rule (Eq. (2.23) in \cite{Bilbao2009}) which is
\begin{equation}
    \dxp (u_lw_l) = (\dxp u_l)(\mup{w_l}) + (\mup u_l)(\dxp w_l).
\end{equation}
In the case of \eqref{eq:discWebster}, $u_l \triangleq \Sm$ and $w_l \triangleq \dxm\Psiln$. Expanding (retaining the notation for $\Sbar$) and solving for $\Psinp$ yields
\begin{equation}
    \begin{aligned}
        \frac{\Sbar}{k^2}(\Psinp - 2\Psiln+\Psinm) &= c^2\left((\dxp \Sm)(\mup \dxm \Psiln) + (\mup \Sm)(\dxp \dxm \Psiln)\right)\\
        \Psinp - 2\Psiln+\Psinm &= \frac{c^2k^2}{\Sbar}\bigg(\frac{1}{h}(\Sp - \Sm)\frac{1}{2h}\overbrace{(\Psilp -\Psilm)}^{\mup\dxm\Psiln = \delta_{x\cdot}\Psiln}\\
        &\qquad\qquad+\frac{1}{2}(\Sp + \Sm)\frac{1}{h^2}(\Psilp-2\Psiln+\Psilm)\bigg)\\
        \Psinp &= 2\Psiln-\Psinm + \overbrace{\frac{\lambda^2}{2\Sbar}}^{\lambda = \frac{ck}{h}}\Big(\Sp\Psilp - \Sp\Psilm - \Sm\Psilp + \Sm\Psilm \\
        &+ \Sp\Psilp + \Sp\Psilm + \Sm\Psilp + \Sm\Psilm - 2 (\Sp + \Sm)\Psiln\Big)\\
        \Psinp &= 2\Psiln-\Psinm+ \frac{\lambda^2}{2\Sbar}\Big(2\Sp\Psilp + 2\Sm\Psilm - 4\Sbar\Psiln\Big)\\
        \Psinp &= 2\Psiln-\Psinm+ \frac{\lambda^2\Sp}{\Sbar}\Psilp + \frac{\lambda^2\Sm}{\Sbar}\Psilm - 2\lambda^2\Psiln\\
        \Psinp &= 2(1-\lambda^2)\Psiln-\Psinm+ \frac{\lambda^2\Sp}{\Sbar}\Psilp + \frac{\lambda^2\Sm}{\Sbar}\Psilm,
    \end{aligned}
\end{equation}
which is identical to Eq. (19.51) in \cite{Bilbao2018}.

\subsection{Boundary Conditions}

\bibliographystyle{plain}
\bibliography{bibliography}

\end{document}
